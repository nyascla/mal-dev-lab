\paragraph{Exfiltración} Durante las actividades de post‑explotación, el atacante ha ido exfiltrando información de los sistemas comprometidos de manera progresiva y sostenida, aprovechando los accesos obtenidos sin generar patrones evidentes que pudieran alertar a los mecanismos de defensa.

Para una descripción detallada del procedimiento empleado por el adversario, consúltese el Capítulo~\ref{chap:red-team}, en particular la Sección~\ref{sec:exfiltracion}.

\begin{leftbar}
    \begin{description}
        \item[Táctica:] Exfiltration (TA0010).
        \item[Técnicas:] Exfiltration Over C2 Channel (T1041).
        \item[Procedimiento:] El implante exfiltró el fichero \texttt{documento\_importante.txt} reutilizando el canal de comunicación C2 ya establecido. La transferencia se realizó encapsulando los datos dentro del tráfico HTTPS del propio implante, evitando generar conexiones adicionales o patrones anómalos que pudieran delatar la actividad.
    \end{description}
\end{leftbar}

\textbf{Artefactos relevantes}
\begin{itemize}
    \item ...
\end{itemize}