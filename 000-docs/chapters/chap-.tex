\chapter{Introducción}

\begin{quote}
\textit{¿Porque alguien deberia leer este trabajo?}
\end{quote}

\section{Adversary emulation}

justificacion: que es y porque es interesante

\section{Objetivos del trabajo}

El objetivo de este trabajo es desarrollar un escenario didáctico que permita a
los analistas de ciberseguridad mejorar sus capacidades de deteccion e investigación
mediante la comprensión del proceso completo de creación de un ciberataque. El
resultado final del proyecto será un conjunto de artefactos forenses de los
sistemas infectados, que servirá como base para ejercicios de análisis forense
(blue team).

Para generar dicho recurso, el núcleo del trabajo se centrará en la concepción,
diseño y desarrollo de un ciberataque realista, su ejecución controlada en un
sistema víctima y la documentación íntegra del proceso ofensivo llevado a cabo. 
De este modo, quien reciba los artefactos forenses podrá analizar la infección 
desde el punto de vista del analista forense, pero también interpretar las acciones del atacante, entendiendo mejor
el cómo y el porqué de cada artefacto encontrado.

En definitiva, este trabajo busca aproximar al analista a la mentalidad del
atacante, con el fin de reforzar sus habilidades de detección, interpretación y
respuesta ante amenazas reales.



\chapter{Marco Teórico}

\begin{quote}
\textit{¿Como la industria consigue tener un lenguaje comun para poder transmitir
informacion sobre adversarios de forma efectiva?}
\end{quote}

\section{MITRE ATT\&CK}

el procedimiento debe ser lo suficientemente detallaco como para que el blue team 
pueda crear una regla de deteccion y el red team pueda recrearlo

\section{Cyber Kill Chain (CKC)}

el framework consiste en dividir un ciberatque en las siguientes fases:
\begin{enumerate}
    \item Reconocimiento: investigacion del objetivo
    \item Armazenamiento: creacion del payload
    \item Entrega: envio del payload al objetivo
    \item Explotacion: ejecucion del payload
    \item Instalacion: sistema de staging
    \item Comando y control: establecimiento del canal de comunicacion
    \item Acciones sobre objetivos: acciones finales
\end{enumerate}

\section{Threat Intelligence}



% 
% 
% 
\chapter{Creando nuestro adversario}

El atacante logra la ejecución de código en la máquina objetivo e inicia el
stage 1. Este primer componente se conecta a un servidor controlado por el
adversario para descargar el stage 2. 

En algunos escenarios, entre ambos
stages, se realizan tareas de reconocimiento interno para recopilar
información sobre la víctima (por ejemplo, enumeración de usuarios, servicios
o topología de red). 

El stage 2 consiste en una DLL maliciosa que se hace
pasar por una biblioteca legítima del sistema. Esta DLL actúa como proxy de
la original, garantizando que las aplicaciones que dependen de ella funcionen
correctamente sin generar errores, mientras que, en paralelo, carga en
memoria un implante de Sliver. 

Una vez inyectado el implante, este ejecuta
acciones de escaneo interno, movimiento lateral y establece comunicación
periódica (beaconing) con el servidor de comando y control (C2). Cuando el
atacante lo considere oportuno, se activará el payload de ransomware,
procediendo al cifrado de archivos en el dispositivo comprometido.

\section{Fase 1. Reconocimiento}
\textit{Ver el análisis detallado en la Sección~\ref{sec:detalle-reconocimiento}.}


\section{Fase 2. Armamento}

\textit{Ver el análisis detallado en la Sección~\ref{sec:detalle-armamento}.}


\section{Fase 3 (Opt 1)): Entrega: BadUSB}

\textit{Ver el análisis detallado en la Sección~\ref{sec:detalle-entrega}.}

\begin{leftbar}
    \begin{description}
        \item[Táctica:] Initial Access (TA0001) \cite{mitre-TA0001}.
        \item[Técnica:] Replication Through Removable Media (T1091) \cite{mitre-T1091}.
        \item[Procedimiento:] Se emplea un dispositivo \textit{BadUSB} que, al ser conectado, emula un dispositivo de interfaz humana (HID) para inyectar y ejecutar un script de PowerShell.
    \end{description}
\end{leftbar}

\section{Fase 3 (Opt 2): Entrega: Buffer vulnerable}

\section{Fase 3 (Opt 3): Entrega: Phishing}

\section{Fase 4: Explotación}

\textit{Ver el análisis detallado en la Sección~\ref{sec:detalle-explotacion}.}

\begin{leftbar}
    \begin{description}
        \item[Táctica:] Execution (TA0002) \cite{mitre-TA0002}.
        \item[Técnica:] Command and Scripting Interpreter: PowerShell (T1059.001) \cite{mitre-T1059-001}.
        \item[Procedimiento:] El script inicial utiliza técnicas de \textit{marshalling} para evadir la Interfaz de Escaneo Antimalware (AMSI) de PowerShell, permitiendo la ejecución de código malicioso en memoria.
    \end{description}
\end{leftbar}

\begin{leftbar}
    \begin{description}
        \item[Táctica:] Command and Control (TA0011) \cite{mitre-TA0011}.
        \item[Técnica:] Ingress Tool Transfer (T1105) \cite{mitre-T1105}.
        \item[Procedimiento:] El \textit{payload} inicial se conecta a un servidor C2 externo para descargar la siguiente fase (\textit{stage 2}).
    \end{description}
\end{leftbar}

\section{Fase 5: Instalación y Persistencia}
 
\textit{Ver el análisis detallado en la Sección~\ref{sec:detalle-instalacion}.}

\begin{leftbar}
    \begin{description}
        \item[Tácticas:]
            \begin{itemize}
                \item Persistence (TA0003)
                \item Defense Evasion (TA0005)
            \end{itemize}
        \item[Técnica:] Hijack Execution Flow: DLL Search Order Hijacking (T1574.001) \cite{mitre-T1574-001}.
        \item[Procedimiento:] La DLL (\textit{stage 2}) se escribe en disco en una ubicación específica para ser cargada por un proceso legítimo y vulnerable a secuestro de DLL, asegurando la persistencia. Se consultan repositorios como \cite{hijacklibs} para identificar binarios vulnerables.
    \end{description}
\end{leftbar}

\begin{leftbar}
    \begin{description}
        \item[Táctica:] Defense Evasion (TA0005).
        \item[Técnica:] Reflective Code Loading (T1620) \cite{mitre-T1620}.
        \item[Procedimiento:] El \textit{stage 2} (la DLL secuestrada) actúa como un cargador (\textit{loader}) que mapea reflectivamente el \textit{payload} final directamente en la memoria del proceso anfitrión, evitando así las llamadas estándar a \texttt{LoadLibrary} y la detección basada en módulos cargados.
    \end{description}
\end{leftbar}

\section{Fase 6: Mando y Control (C2)}

\textit{Ver el análisis detallado en la Sección~\ref{sec:detalle-c2}.}

\begin{leftbar}
    \begin{description}
        \item[Táctica:] Command and Control (TA0011) \cite{mitre-TA0011}.
        \item[Técnicas:]
            \begin{itemize}
                \item Application Layer Protocol: Web Protocols (T1071.001) \cite{mitre-T1071-001}.
                \item Encrypted Channel (T1573) \cite{mitre-T1573}.
            \end{itemize}
        \item[Procedimiento:] El implante final establece un canal de comunicación (\textit{beaconing}) con el servidor C2. Esta comunicación se tuneliza sobre HTTPS (T1071.001) y utiliza un canal de cifrado (T1573) para ocultar las órdenes y la exfiltración de datos, mimetizando el tráfico web legítimo.
    \end{description}
\end{leftbar}

\section{Fase 7: Acciones sobre el Objetivo}

\textit{Ver el análisis detallado en la Sección~\ref{sec:detalle-acciones}.}

\begin{leftbar}
    \begin{description}
        \item[Táctica:] Impact (TA0040) \cite{mitre-TA0040}.
        \item[Técnica:] Data Encrypted for Impact (T1486) \cite{mitre-T1486}.
        \item[Procedimiento:] El \textit{payload} final ejecuta su objetivo principal: cifrar los archivos del sistema de la víctima y solicitar un rescate para su recuperación.
    \end{description}
\end{leftbar}

% 
% 
% 
\chapter{Analizando al adversario}

\section{Análisis del Reconocimiento}
\label{sec:detalle-reconocimiento} 

\section{Análisis del Armamento}
\label{sec:detalle-armamento} 

\section{Análisis de la Entrega}
\label{sec:detalle-entrega} 

\section{Análisis de la Explotación}
\label{sec:detalle-explotacion} 

\section{Análisis de la Instalación y Persistencia}
\label{sec:detalle-instalacion} 

\section{Análisis del C2}
\label{sec:detalle-c2} 

\section{Análisis de las acciones sobre el objetivo}
\label{sec:detalle-acciones} 