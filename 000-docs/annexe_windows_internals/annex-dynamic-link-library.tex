\section{DLL (Dynamic-Link Library)}
En Windows, un proceso puede depender de múltiples librerías dinámicas 
(\texttt{DLLs}) para ejecutar ciertas funcionalidades. Estas DLLs se cargan en 
el espacio de direcciones del proceso y su interacción se realiza a través de 
estructuras específicas que Windows mantiene en memoria.

\subsection{Carga de DLLs en un proceso}
Cuando un proceso arranca, indica qué DLLs necesita y qué funciones de cada DLL 
va a utilizar. Windows realiza los siguientes pasos:

\begin{enumerate}
    \item Verifica si la DLL ya está cargada en memoria; si no lo está, la carga 
    en el espacio de direcciones del proceso.
    \item Si ya existe en memoria, se reutiliza la misma copia en memoria física, 
    pero cada proceso mantiene su propio mapeo en su espacio de direcciones virtuales.
    \item Localiza dentro de la DLL la función solicitada por el proceso.
    \item Proporciona al proceso la dirección en memoria de la función, de 
    manera que cada llamada a esa función realmente salta a la dirección 
    correspondiente dentro de la DLL cargada.
\end{enumerate}

Cada DLL se carga como un módulo completo, incluyendo sus secciones 
\texttt{.text}, \texttt{.rdata}, \texttt{.data}, etc., mapeadas en memoria 
contigua a partir de la \emph{base address} del módulo.

\subsection{Import Address Table (IAT)}
Para que un proceso llame a funciones de una DLL, Windows utiliza la 
\textbf{Import Address Table (IAT)}. La IAT contiene punteros a las funciones 
importadas por el proceso. Durante la carga:

\begin{enumerate}
    \item Windows localiza cada función solicitada dentro de la DLL 
    correspondiente.
    \item Escribe en la IAT la dirección en memoria de la función.
\end{enumerate}

Cuando el proceso ejecuta una llamada a una función importada, en realidad está 
saltando a la dirección contenida en la IAT, que apunta a la DLL cargada en 
memoria.

\subsection{Resumen}
En conjunto, el PEB y la IAT permiten a un proceso interactuar de manera 
eficiente con sus DLLs:

\begin{itemize}
    \item El PEB mantiene un registro de todos los módulos cargados y sus 
    direcciones.
    \item La IAT traduce las llamadas a funciones de la DLL a direcciones 
    concretas en memoria.
\end{itemize}

Esto garantiza que el proceso pueda utilizar código externo (DLLs) sin necesidad 
de incorporarlo estáticamente en su binario, manteniendo modularidad y eficiencia 
en el uso de memoria.







