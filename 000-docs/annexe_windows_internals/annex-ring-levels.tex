\section{User Mode y Kernel Mode}

El espacio de ejecución en Windows se divide en dos niveles principales: \textbf{User Mode} y \textbf{Kernel Mode}. Cada uno tiene responsabilidades y privilegios distintos.  

En este ejemplo se muestran todas las llamadas que realiza una aplicación para que una función pase de User Mode a Kernel Mode, es decir, el flujo completo desde las librerías de alto nivel hasta la ejecución de operaciones críticas en el núcleo del sistema.

\begin{enumerate}
    \item \textbf{User Mode (U) – Aplicaciones y librerías de alto nivel}
    \begin{itemize}
        \item \texttt{notepad++.exe}: operaciones internas de la aplicación, \\ como: \texttt{SetLibraryProperty}.
        \item \texttt{user32.dll}: funciones de interacción con la interfaz de Windows, por ejemplo \texttt{CallWindowProcW} o \texttt{IsWindowUnicode}.
        \item \texttt{KernelBase.dll}: funciones de usuario de alto nivel que abstraen operaciones del sistema, como \texttt{WriteFile + 0x8d}.
        \item \texttt{ntdll.dll}: proporciona la interfaz directa hacia las syscalls, por ejemplo \texttt{ZwWriteFile}.
    \end{itemize}

    \item \textbf{Kernel Mode (K) – Operaciones de bajo nivel y acceso a hardware}
    \begin{itemize}
        \item \texttt{FLTMGR.SYS}: manejo de filtros de sistema de archivos, con funciones como \texttt{FltPerformSynchronousIo} y \texttt{FltSetCancelCompletion}.
        \item \texttt{ntoskrnl.exe}: operaciones sobre drivers y dispositivos, incluyendo \texttt{IoCallDriver}, \texttt{NtDeviceIoControlFile} y \texttt{NtWriteFile}.
    \end{itemize}
\end{enumerate}

El flujo típico de ejecución comienza en User Mode, con la aplicación y librerías de alto nivel, y termina en Kernel Mode, donde se realizan operaciones críticas sobre el sistema y el hardware, como la escritura en disco con privilegios elevados.
Este ejemplo ilustra cómo cada llamada permite que una función pase de User Mode a Kernel Mode, siguiendo la cadena completa de llamadas hasta el núcleo del sistema.


\begin{center}
    \includegraphics[width=0.95\textwidth]{figures/fig-syscall-path.png}
    
    \vspace{0.3cm}
    \small{Figura: Captura de la herramienta \textit{Process Monitor (ProMon)} de Sysinternals mostrando el flujo de llamadas de User Mode a Kernel Mode.}
\end{center}
