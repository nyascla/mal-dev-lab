\section{Memoria Virtual}

\subsection{Problemas Memoria Fisica}
\begin{itemize}
    \item Escasez de memoria
    \item Fragmentación
    \item Acceso a memoria (Seguridad)
\end{itemize}

\subsection{Acceso indirecto a memoria}

Cada proceso tiene su propio espacio de direcciones virtual, 
Para el proceso es como tener un espacio de momoria fisica contiguo reservado,
Pero por detras el SO esta mapeando esas direcciones virtuales a distintas regiones de memoria fisica.

\subsection{Paginación}

Mapear cada byte de la memoria virtual a un byte de la moria fisica seria muy ineficiente,
Se crean paginas de moria que son bloques de bytes para simplificar este proceso (normalmente de 4KB | 4096 bytes),

page - memoria virtual
frame - memoria fisica

Cuando un proceso se carga sus paginas son cargadas en frames disponibles en moria fisica

\subsection{Tabla de paginas}

Cada proceso tiene una tabla de paginas que mapea las direcciones virtuales a las fisicas,
La tabla de paginas es gestionada por la MMU (Memory Management Unit),
Para cada pagina se mantiene, su numero de frame, permisos, etc. 

\subsection{MMU}

La CPU tiene que accedr a la RAM primero para consultar la tabla de paginasy Despues para acceder a la memoria fisica, 
para cada lectura tenemos que hacer 2 accesos a memoria,
Para optimizar esto, la MMU, hardaware especifico de la CPU, se encarga de traducir las direcciones virtuales a fisicas,
la MMU tiene un caché de la tabla de paginas llamado TLB (Translation Lookaside Buffer),
La MMU tambien es la encargada de gestion que un proceso no acceda a memoria que no le pertenece,

\subsection{Acceso a memoria}

Cuando un proceso quiere acceder a una dirección de memoria, la CPU consulta la MMU,
En la MMU la direccon se divide en dos partes, el numero de pagina y el offset dentro de la pagina,

Pege number - los bits mas significativos de la direccion virtual
Page offset - los bits menos significativos de la direccion virtual

Ejemplo:
En una pagina de 4KB (4096 bytes), el offset ocupa los 12 bits menos significativos,
El resto de bits son los mas significativos y representan el numero de pagina,
