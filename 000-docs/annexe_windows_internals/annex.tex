\chapter{Anexo A: Windows Internals}

En este anexo se ofrece una visión sintetizada de varios componentes fundamentales de Windows Internals que resultan esenciales para comprender el comportamiento del malware y de las técnicas de evasión modernas. Se introducen los conceptos de memoria virtual, el formato ejecutable PE y la arquitectura interna de los procesos en Windows, proporcionando el contexto técnico necesario para interpretar adecuadamente las operaciones realizadas por el adversario durante el ataque. Junto a cada explicación teórica, se incluyen pequeños fragmentos de código en C que muestran cómo navegar e interactuar con estas estructuras en tiempo de ejecución, facilitando una comprensión práctica de su funcionamiento real dentro del sistema operativo.

\section{Memoria Virtual}

\subsection{Problemas Memoria Fisica}
\begin{itemize}
    \item Escasez de memoria
    \item Fragmentación
    \item Acceso a memoria (Seguridad)
\end{itemize}

\subsection{Acceso indirecto a memoria}

Cada proceso tiene su propio espacio de direcciones virtual, 
Para el proceso es como tener un espacio de momoria fisica contiguo reservado,
Pero por detras el SO esta mapeando esas direcciones virtuales a distintas regiones de memoria fisica.

\subsection{Paginación}

Mapear cada byte de la memoria virtual a un byte de la moria fisica seria muy ineficiente,
Se crean paginas de moria que son bloques de bytes para simplificar este proceso (normalmente de 4KB | 4096 bytes),

page - memoria virtual
frame - memoria fisica

Cuando un proceso se carga sus paginas son cargadas en frames disponibles en moria fisica

\subsection{Tabla de paginas}

Cada proceso tiene una tabla de paginas que mapea las direcciones virtuales a las fisicas,
La tabla de paginas es gestionada por la MMU (Memory Management Unit),
Para cada pagina se mantiene, su numero de frame, permisos, etc. 

\subsection{MMU}

La CPU tiene que accedr a la RAM primero para consultar la tabla de paginasy Despues para acceder a la memoria fisica, 
para cada lectura tenemos que hacer 2 accesos a memoria,
Para optimizar esto, la MMU, hardaware especifico de la CPU, se encarga de traducir las direcciones virtuales a fisicas,
la MMU tiene un caché de la tabla de paginas llamado TLB (Translation Lookaside Buffer),
La MMU tambien es la encargada de gestion que un proceso no acceda a memoria que no le pertenece,

\subsection{Acceso a memoria}

Cuando un proceso quiere acceder a una dirección de memoria, la CPU consulta la MMU,
En la MMU la direccon se divide en dos partes, el numero de pagina y el offset dentro de la pagina,

Pege number - los bits mas significativos de la direccion virtual
Page offset - los bits menos significativos de la direccion virtual

Ejemplo:
En una pagina de 4KB (4096 bytes), el offset ocupa los 12 bits menos significativos,
El resto de bits son los mas significativos y representan el numero de pagina,

\newpage
\section{Portable Executable (PE)}

El formato \textit{Portable Executable} (PE) es el estándar de Windows para almacenar ejecutables, bibliotecas dinámicas (\textit{DLLs}) y otros tipos de archivos que pueden ser cargados y ejecutados por el sistema operativo.  
Es el equivalente funcional del formato ELF en Linux y proporciona toda la información necesaria para que el cargador de Windows gestione la ejecución de un programa: estructura de secciones, tablas de importación y exportación, encabezados, y metadatos de la aplicación.  
Gracias a esta organización, el sistema puede ubicar código y datos en memoria, resolver dependencias dinámicas y garantizar la correcta ejecución de procesos de manera segura y eficiente.

Los PE file es un fichero con valores hexadecimales dispuestos en un layout especifico para poder ser recorrido

Un archivo PE (\textit{Portable Executable}) es un fichero binario cuyos valores están organizados en un diseño específico que permite recorrerlo y analizarlo de manera estructurada.  
La disposición de los encabezados, secciones y demás campos, incluyendo los offsets estándar de cada componente, se ilustra detalladamente en la infografía disponible en \cite{pe_format_layout_2025}.

\subsection{Headers}

En el offset \texttt{0x00} se encuentran los encabezados del PE, que contienen información general sobre el ejecutable y permiten al sistema operativo interpretar correctamente el archivo. Entre los campos más relevantes se incluyen:

\begin{itemize}
    \item \textbf{AddressOfEntryPoint:} indica la dirección virtual donde comienza la ejecución del programa dentro de la sección de código principal.
    \item \textbf{Machine:} especifica la arquitectura objetivo del ejecutable (por ejemplo, x86 o x86\_64).
    \item \textbf{NumberOfSections:} indica cuántas secciones contiene el PE.
    \item \textbf{DataDirectory:} tabla de referencias a otras estructuras del PE, como la tabla de importaciones, exportaciones, recursos y la tabla de relocaciones.
\end{itemize}

Estos campos permiten al sistema operativo localizar las distintas secciones del PE y gestionar correctamente la memoria virtual del proceso al cargar el ejecutable. La estructura completa y los offsets de cada campo se pueden consultar en la infografía de referencia~\cite{pe_format_layout_2025}.

Implementación en C para iterar y recuperar los campos de los encabezados de un archivo PE~\cite{walking_PE_headers}.

\subsection{Headers Tables}

Las cabeceras de un archivo PE contienen varias tablas que son fundamentales para la correcta creación y carga de un proceso a partir de un ejecutable.  
Estas tablas, que incluyen información sobre exportaciones, importaciones y relocaciones, se encuentran referenciadas en el arreglo \textit{Data Directory} del encabezado opcional del PE.

Para ver cómo recorrer un PE y acceder a cada una de estas tablas, se incluyen ejemplos de código en C en los repositorios de GitHub correspondientes, los cuales se adjuntan en cada sección específica.

\subsubsection{Exports}

Si el archivo PE corresponde a una biblioteca compartida (\textit{DLL}) y exporta funcionalidad, 
esta información se almacena en la tabla de exportaciones.  
A diferencia de Linux, donde las funciones suelen estar disponibles por defecto, en Windows 
cada símbolo debe exportarse explícitamente para poder ser utilizado por otros módulos.

La tabla de exportaciones es esencial durante la carga de un proceso, ya que el \textit{Windows Loader} 
la utiliza para resolver las direcciones de las funciones exportadas por cada DLL, garantizando que puedan 
ser correctamente mapeadas y enlazadas en el espacio de direcciones del proceso que las consume.

Ejemplos de código para analizar y recorrer la tabla de exportaciones de un fichero PE, 
identificando las funciones exportadas y sus direcciones virtuales, pueden encontrarse en~\cite{walking_PE_exports}.  

Además, se incluye una implementación propia de \texttt{GetProcAddress} en ensamblador x64, 
donde se recorre manualmente la tabla de exportaciones del módulo cargado en memoria para localizar 
la dirección real de una función exportada~\cite{custom_getprocaddress_x64_2025}, así como su versión 
equivalente para arquitecturas x86~\cite{custom_getprocaddress_x86_2025}.

\subsubsection{Imports}

La tabla de importaciones es fundamental para el \textit{Windows Loader}, ya que contiene información sobre todas las dependencias externas que un ejecutable requiere. Durante la carga del proceso, el loader consulta esta tabla para determinar qué DLLs deben cargarse en memoria y, una vez cargadas, registra las direcciones de memoria de las funciones importadas.

De esta manera, cada llamada a una función importada en tiempo de ejecución puede resolverse inmediatamente usando la dirección ya conocida, evitando la necesidad de búsquedas adicionales. Esto permite que el código realice llamadas directas a las funciones de las bibliotecas externas de forma eficiente y segura.

Se proporciona una implementación en C para recorrer la tabla de importaciones de un PE y listar las librerías y funciones utilizadas~\cite{walking_PE_imports}.  

Además, se incluye una implementación propia de un \textit{Windows Loader} simplificado para cargar una DLL, en la que puede observarse explícitamente el proceso de parcheo de la \textit{Import Address Table} (IAT) durante la resolución de dependencias~\cite{walkingloader}.

\subsubsection{Relocations}

La tabla de \textit{relocations} es fundamental cuando está habilitado el \texttt{ASLR} (Address Space Layout Randomization).  
Si el ejecutable no puede cargarse en su \textit{ImageBase} preferido, el \textit{Windows Loader} debe reubicarlo en una dirección distinta. Para ello, aplica un \textit{delta} de reubicación calculado como:
\[
\Delta = \text{ImageBaseReal} - \text{ImageBasePreferida}
\]
Este delta debe sumarse a todas las direcciones absolutas del ejecutable, y la lista exacta de ubicaciones que requieren dicho parcheo se encuentra en la sección \texttt{.reloc} del PE.

La tabla de relocaciones no es una tabla plana, sino una lista de \textit{bloques}.  
Cada bloque describe una página de memoria de 4\,KB que contiene una o más direcciones que deben ser ajustadas. 

Cada bloque inicia con esta cabecera de 8 bytes:

\begin{itemize}
    \item \textbf{VirtualAddress (DWORD)}: dirección base de la página afectada.
    \item \textbf{SizeOfBlock (DWORD)}: tamaño total del bloque; el último bloque puede identificarse porque su valor es 0.
\end{itemize}

Tras el encabezado de cada bloque comienza un array de valores \texttt{WORD} (16 bits), donde cada entrada describe un parche. Cada palabra está empaquetada de la siguiente forma:

\begin{itemize}
    \item \textbf{4 bits superiores (Type)}: indican el tipo de relocalización. Ejemplos:
        \begin{itemize}
            \item \texttt{IMAGE\_REL\_BASED\_DIR64}: punteros absolutos de 64 bits.
            \item \texttt{IMAGE\_REL\_BASED\_HIGHLOW}: punteros absolutos de 32 bits.
        \end{itemize}
    \item \textbf{12 bits inferiores (Offset)}: desplazamiento dentro de la página.
\end{itemize}

El uso de un offset de 12 bits se debe a que cada bloque representa exactamente una página de 4\,KB.  
Como una página tiene 4096 bytes y:
\[
2^{12} = 4096,
\]
los 12 bits inferiores son suficientes para direccionar cualquier byte dentro de dicha página.  
La dirección final a parchear se obtiene sumando este offset al \texttt{VirtualAddress} del bloque.

En conjunto, la tabla de relocaciones actúa como un “mapa de parches” generado por el compilador, que permite al \textit{Windows Loader} ajustar todas las direcciones absolutas del binario cuando se carga en una posición distinta a la prevista originalmente.

Ejemplo de código para iterar la tabla de relocaciones de un PE y analizar los desplazamientos aplicables a las direcciones del ejecutable~\cite{walking_PE_relocs}.

\subsection{Sections}

En términos de funcionalidad, las secciones del PE son equivalentes a los segmentos de un archivo ELF en Linux.

Las secciones se encuentran inmediatamente después de los encabezados del PE y están representadas como un conjunto de estructuras, una por cada sección.  
Cada estructura proporciona información como el offset y el tamaño de la sección, tanto en disco como en memoria. Esta información es utilizada por el \textit{Windows Loader} para definir las distintas regiones de memoria con sus protecciones específicas (lectura, escritura, ejecución).

Las secciones de un proceso en memoria, como \texttt{.text}, \texttt{.data}, \texttt{.reloc}, \texttt{.idata} o \texttt{.edata}, ya están definidas en el archivo PE.  
Durante la carga, el loader recorre estas secciones, las mapea a las direcciones que tendrán en memoria y aplica los parcheos necesarios, como completar la IAT con las direcciones reales de las funciones importadas y ajustar cualquier relocación requerida.

En el siguiente código se muestra cómo recorrer las secciones de un archivo PE~\cite{walking_PE_headers}.

En la implementación propia de un \textit{Windows Loader} simplificado~\cite{walkingloader}, también se muestra cómo, desde C, se pueden iterar las secciones del PE, reservar la memoria correspondiente, mapearlas adecuadamente y aplicar los parcheos necesarios.


\subsection{Ejemplo práctico: parcheo de un PE y modificación del Entry Point}

Como cierre a esta sección, se presenta un ejemplo práctico que ilustra cómo modificar un archivo PE para alterar su flujo de ejecución durante la carga. En este proyecto se incluye un script en Python que automatiza el proceso de parcheo de un binario arbitrario, incorporando un \textit{shellcode} desarrollado específicamente para esta prueba. Dicho \textit{shellcode} muestra una ventana mediante la función \texttt{MessageBoxA} antes de que el programa continúe ejecutándose con normalidad.

La técnica empleada consiste en insertar el \textit{shellcode} dentro del propio archivo PE, ubicándolo en una zona del fichero que no contenga datos relevantes, y modificar el \textit{Entry Point} para que apunte a este nuevo código en lugar del original. Para mantener el comportamiento legítimo del ejecutable, el \textit{shellcode} incluye un salto de retorno hacia la dirección del \textit{Entry Point} original, de manera que, tras finalizar la ejecución de la rutina inyectada, el programa retoma su flujo normal sin alteraciones perceptibles.

Este ejemplo demuestra de forma práctica cómo las estructuras internas del formato PE pueden manipularse para modificar el inicio de ejecución sin comprometer la funcionalidad del binario, proporcionando un entorno ideal para experimentar con técnicas de parcheo, carga manual y redirección de control.

El script encargado del parcheo puede encontrarse en~\cite{binary_patching_entrypoint},
y el \textit{shellcode} desarrollado específicamente para esta prueba en~\cite{messagebox_patch_shellcode}.
\newpage
\section{Procesos Windows}

Un proceso en Windows representa una entidad de ejecución que actúa como contenedor de los recursos necesarios para que uno o varios hilos puedan ejecutarse. Desde el punto de vista del sistema operativo, un proceso no ejecuta código por sí mismo, sino que proporciona el contexto y los recursos que los hilos utilizan.  
A continuación se describen sus componentes principales:

\begin{itemize}
    \item \textbf{EPROCESS:}  
    Estructura interna del kernel que representa a un proceso. Contiene información crítica como el identificador del proceso (PID), la lista de hilos asociados, permisos, límites de memoria, una referencia al \textit{Process Environment Block} (PEB), así como metadatos utilizados por el planificador y el gestor de seguridad. Constituye la representación fundamental del proceso en modo kernel.

    \item \textbf{Hilos:}  
    Conjunto de hilos pertenecientes al proceso. Cada hilo es administrado en modo kernel mediante la estructura \texttt{ETHREAD}, mientras que en modo usuario dispone de su propio \textit{Thread Environment Block} (TEB).

    \item \textbf{Handles:}  
    Los \textit{handles} son referencias a objetos administrados por el kernel, tales como archivos, sockets, secciones de memoria, procesos o hilos. Funcionan de manera equivalente a los \textit{file descriptors} en Linux, pero con un modelo mucho más generalista. El proceso mantiene una \textit{handle table} que almacena y gestiona estas referencias.

    \item \textbf{Memoria:}  
    Comprende todo el espacio de direcciones virtuales asignado al proceso, incluyendo secciones de código, datos, pilas, \textit{heaps} y regiones mapeadas dinámicamente.

    \item \textbf{Módulos (DLLs):}  
    Lista de módulos y bibliotecas cargadas en el proceso, mantenida en el \textit{PEB Ldr}.
\end{itemize}


\subsection{Creación de un proceso (Kernel)}

La creación de un proceso en Windows sigue una secuencia estricta de pasos a nivel del kernel.

\subsubsection*{(1) Inicialización del espacio de direcciones}

\begin{itemize}[leftmargin=1.5cm]

    \item \textbf{Modelo en Linux vs.\ Windows:}  
    En Linux, los procesos se crean mediante \texttt{fork()}.  
    En Windows el kernel crea un proceso completamente nuevo, partiendo de un espacio de direcciones vacío que debe ser configurado desde el inicio.

    \item \textbf{Mapeo de \texttt{KUSER\_SHARED\_DATA}:}  
    Una de las primeras acciones del kernel es mapear en el nuevo espacio de direcciones una página especial de 4\,KB llamada \texttt{KUSER\_SHARED\_DATA}.  
    Esta página está presente simultáneamente en \textit{user mode} y \textit{kernel mode} y permite compartir información sin necesidad de realizar una llamada al sistema.  
    Contiene datos como:
    \begin{itemize}
        \item El reloj del sistema,
        \item Información de versión y arquitectura,
        \item Rutas internas del sistema operativo,
        \item Indicadores de configuración global.
    \end{itemize}

    \item \textbf{Mapeo inicial del ejecutable:}  
    Antes de cargar \texttt{ntdll.dll}, el kernel reserva el espacio de direcciones del proceso y mapea únicamente la sección de código principal (\texttt{.text}) del ejecutable en memoria.  
    
    \item \textbf{Mapeo de \texttt{ntdll.dll}:}  
    El siguiente módulo en cargarse es \texttt{ntdll.dll}.  
    Este componente es esencial, ya que:
    \begin{itemize}
        \item Contiene las \textit{syscall stubs} que permiten la transición a \textit{kernel mode};
        \item Implementa funciones internas utilizadas por el loader de \textit{user mode};
        \item Cumple un rol análogo a \texttt{ld.so} en sistemas Linux.
    \end{itemize}

    \item \textbf{Asignación del PEB (Process Environment Block):}  
    Una vez inicializado el espacio de direcciones, el kernel crea y mapea el \texttt{PEB}, una estructura de 1--2 páginas ubicada en \textit{user mode} que describe el estado general del proceso.  
    El \texttt{PEB} contiene, entre otros:
    \begin{itemize}
        \item Variables de entorno,
        \item El directorio de trabajo,
        \item La lista de módulos cargados (\texttt{PEB.Ldr}),
        \item Punteros al \textit{heap} y datos relacionados a la pila,
        \item La dirección base de la imagen del ejecutable.
    \end{itemize}

\end{itemize}

\subsubsection*{(2) Creación del hilo inicial}

\begin{itemize}[leftmargin=1.5cm]

    \item \textbf{Mapeo del stack:}  
    El kernel reserva y mapea la pila (\textit{stack}) del hilo principal con sus límites superior e inferior, además de aplicar las protecciones apropiadas.

    \item \textbf{Mapeo del TEB (Thread Environment Block):}  
    Para cada hilo se asigna un \texttt{TEB}, una estructura de aproximadamente 2 páginas que almacena información específica del hilo.

    \item \textbf{Inicialización del punto de entrada del hilo:}  
    Finalmente, el contador de programa (IP) del nuevo hilo se inicializa en la función  
    \texttt{ntdll!LdrInitializeThunk}.  
    Esta rutina es responsable de completar la carga dinámica del proceso en modo usuario, incluyendo:
    \begin{itemize}
        \item Resolución de importaciones,
        \item Ejecución de inicializadores (\texttt{TLS callbacks}),
        \item La transferencia final al \textit{entry point} del ejecutable.
    \end{itemize}

\end{itemize}


\subsection{Process Environment Block (PEB)}

El \textbf{Process Environment Block (PEB)} es una estructura interna de Windows que almacena información esencial sobre un proceso en ejecución. Aunque Microsoft no publica una definición oficial completa de esta estructura, su diseño es conocido gracias al análisis inverso y puede variar ligeramente entre versiones del sistema operativo. El PEB reside en \textit{user mode} y es único para cada proceso.

Entre los elementos más relevantes del PEB se encuentran:

\begin{itemize}
    \item \textbf{ImageBaseAddress:} Dirección base donde se ha mapeado la imagen principal del proceso.
    \item \textbf{BeingDebugged:} Indicador de si el proceso está siendo depurado.
    \item \textbf{Ldr (PEB\_LDR\_DATA):} Estructura que mantiene la lista de módulos cargados (DLLs).  
    Recorrer esta lista es un mecanismo ampliamente utilizado por \textit{shellcode} para recuperar la base de \texttt{kernel32.dll} y resolver funciones de la WinAPI.
\end{itemize}

Una explicación detallada y ejemplos de cómo iterar el PEB, tanto desde C como desde WinDbg, pueden encontrarse en \cite{nyascla_walking_peb}.

\subsubsection*{Iterar el PEB para recuperar la base de un módulo}

El procedimiento estándar para recorrer el PEB y obtener la base de una DLL concreta es el siguiente:

\begin{enumerate}
    \item Desde la estructura \texttt{\_PEB}, acceder al miembro \texttt{Ldr}, de tipo \texttt{\_PEB\_LDR\_DATA}.
    \item Desde \texttt{\_PEB\_LDR\_DATA}, acceder a la lista doblemente enlazada \texttt{InLoadOrderModuleList}, de tipo \texttt{\_LIST\_ENTRY}.
    \item Cada entrada de esta lista corresponde a una estructura \texttt{\_LDR\_DATA\_TABLE\_ENTRY}.
    \item Desde \texttt{\_LDR\_DATA\_TABLE\_ENTRY}, acceder al campo \texttt{BaseDllName}, de tipo \texttt{\_UNICODE\_STRING}, para obtener el nombre del módulo.
    \item Si el nombre coincide con el módulo deseado, recuperar el campo \texttt{DllBase} desde \texttt{\_LDR\_DATA\_TABLE\_ENTRY}, que contiene la dirección base cargada en memoria.
    \item Si no coincide, avanzar al siguiente elemento de la lista (offset \texttt{0x00} de \texttt{\_LDR\_DATA\_TABLE\_ENTRY}).
\end{enumerate}

Implementaciones prácticas de esta técnica en ensamblador, desarrolladas para este proyecto, están disponibles tanto para \textbf{x64} como para \textbf{x86}:

\begin{itemize}
    \item Versión x64: \cite{nyascla_get_module_handle_x64}
    \item Versión x86: \cite{nyascla_get_module_handle_x86}
\end{itemize}

\subsection{Secciones de un proceso}

El espacio de direcciones de un proceso en memoria se organiza en varias secciones, que se enumeran de las direcciones más altas a las más bajas:

\begin{description}

    \item[\textbf{Páginas del Kernel}]  
    Contienen el kernel y ocupan las mismas direcciones en todos los procesos.  
    Como todas apuntan a los mismos \textit{frames} de memoria física, el kernel no se carga repetidamente en cada proceso.  
    Solo se accede a ellas mediante llamadas al sistema.

    \item[\textbf{Stack (RW)}]  
    Área de almacenamiento \textit{LIFO} que crece hacia abajo, desde el final del espacio de direcciones hacia la BSS.  
    Cada función tiene su propio \textit{stack frame} con variables locales y argumentos.  
    Su tamaño suele estar limitado entre 1 y 8 MB, y su acceso mediante el puntero de pila es muy rápido.

    \item[\textbf{Heap (RW)}]  
    Espacio para memoria dinámica, gestionada con llamadas como \texttt{malloc}.  
    Crece hacia arriba, desde el final de la BSS hasta el inicio del stack.  
    Su administración es más lenta que la del stack.

    \item[\textbf{BSS y Data (RW)}]  
    Contienen variables globales y estáticas.  
    La sección \textbf{BSS} guarda variables no inicializadas, mientras que \textbf{Data} almacena variables inicializadas.

    \item[\textbf{Text (RX)}]  
    Contiene el código ejecutable del programa, cargado desde el binario en disco.  
    Generalmente es de solo lectura y ejecución.

\end{description}

\section{Genealogía de Procesos en Windows}

Al iniciar, Windows lanza una serie de procesos fundamentales para gestionar todo el sistema operativo, desde la administración de recursos y servicios hasta la interacción con el usuario. Estos procesos forman una jerarquía, donde cada nuevo proceso puede generar otros procesos hijos según las necesidades del sistema o del usuario.

La genealogía de procesos muestra esta estructura desde el arranque del sistema hasta el inicio de sesión del usuario. Comprenderla es clave para análisis forense, respuesta a incidentes y detección de malware, ya que permite identificar comportamientos anómalos en los procesos del sistema.

\vspace{1em}
\noindent Una genealogía típica de procesos en Windows se representa a continuación:

\begin{center}
    \includegraphics[width=0.95\textwidth]{figures/fig-annex-win-process-genealogy.png}
\end{center}

\noindent Fuente: \texttt{youtube.com/13cubed}

\subsection*{Nivel 0: Proceso raíz}
\begin{itemize}
    \item \textbf{System}: Primer proceso de espacio de usuario iniciado por el kernel. Tiene un PID fijo (normalmente 4) y no tiene padre.
\end{itemize}

\subsection*{Nivel 1: Session Manager}
\begin{itemize}
    \item \textbf{smss.exe (Session Manager Subsystem)}: Primer proceso real del espacio de usuario. Inicia las sesiones del sistema y lanza procesos críticos como \texttt{csrss.exe}, \texttt{wininit.exe} y \texttt{winlogon.exe}.
\end{itemize}

\subsection*{Nivel 2: Procesos críticos del sistema}
\begin{itemize}
    \item \textbf{csrss.exe}: Gestiona la consola y la creación de procesos. Hay una instancia por sesión.
    \item \textbf{wininit.exe}: Lanza procesos esenciales como \texttt{services.exe}, \texttt{lsaiso.exe} y \texttt{lsass.exe}.
    \item \textbf{winlogon.exe}: Gestiona la autenticación del usuario y permanece activo durante la sesión interactiva.
\end{itemize}

\subsection*{Nivel 3: Procesos del sistema}
\begin{itemize}
    \item \textbf{services.exe}: Service Control Manager. Inicia y gestiona los servicios del sistema, incluidos los alojados por \texttt{svchost.exe}.
    \item \textbf{lsaiso.exe}: Proceso de seguridad aislado que implementa funciones de cifrado y autenticación (opcional en versiones modernas de Windows).
    \item \textbf{lsass.exe}: Local Security Authority Subsystem Service. Encargado de políticas de seguridad, autenticación y gestión de credenciales.
\end{itemize}

\subsection*{Nivel 4: Servicios alojados y utilidades del sistema}
\begin{itemize}
    \item \textbf{svchost.exe}: Proceso contenedor que aloja múltiples servicios del sistema. Existen varias instancias según los grupos de servicios.
    \item \textbf{runtimebroker.exe / taskhostw.exe}: Procesos auxiliares para la ejecución de aplicaciones y tareas programadas.
\end{itemize}

\subsection*{Procesos del usuario interactivo}
\begin{itemize}
    \item \textbf{userinit.exe}: Iniciado por \texttt{winlogon.exe} tras la autenticación. Lanza el shell principal del usuario y termina.
    \item \textbf{explorer.exe}: Shell gráfico de Windows, representa el escritorio, la barra de tareas y el menú de inicio. Es el proceso raíz del entorno del usuario.
\end{itemize}


\newpage
\section{DLL (Dynamic-Link Library)}
En Windows, un proceso puede depender de múltiples librerías dinámicas 
(\texttt{DLLs}) para ejecutar ciertas funcionalidades. Estas DLLs se cargan en 
el espacio de direcciones del proceso y su interacción se realiza a través de 
estructuras específicas que Windows mantiene en memoria.

\subsection{Carga de DLLs en un proceso}
Cuando un proceso arranca, indica qué DLLs necesita y qué funciones de cada DLL 
va a utilizar. Windows realiza los siguientes pasos:

\begin{enumerate}
    \item Verifica si la DLL ya está cargada en memoria; si no lo está, la carga 
    en el espacio de direcciones del proceso.
    \item Si ya existe en memoria, se reutiliza la misma copia en memoria física, 
    pero cada proceso mantiene su propio mapeo en su espacio de direcciones virtuales.
    \item Localiza dentro de la DLL la función solicitada por el proceso.
    \item Proporciona al proceso la dirección en memoria de la función, de 
    manera que cada llamada a esa función realmente salta a la dirección 
    correspondiente dentro de la DLL cargada.
\end{enumerate}

Cada DLL se carga como un módulo completo, incluyendo sus secciones 
\texttt{.text}, \texttt{.rdata}, \texttt{.data}, etc., mapeadas en memoria 
contigua a partir de la \emph{base address} del módulo.

\subsection{Import Address Table (IAT)}
Para que un proceso llame a funciones de una DLL, Windows utiliza la 
\textbf{Import Address Table (IAT)}. La IAT contiene punteros a las funciones 
importadas por el proceso. Durante la carga:

\begin{enumerate}
    \item Windows localiza cada función solicitada dentro de la DLL 
    correspondiente.
    \item Escribe en la IAT la dirección en memoria de la función.
\end{enumerate}

Cuando el proceso ejecuta una llamada a una función importada, en realidad está 
saltando a la dirección contenida en la IAT, que apunta a la DLL cargada en 
memoria.

\subsection{Resumen}
En conjunto, el PEB y la IAT permiten a un proceso interactuar de manera 
eficiente con sus DLLs:

\begin{itemize}
    \item El PEB mantiene un registro de todos los módulos cargados y sus 
    direcciones.
    \item La IAT traduce las llamadas a funciones de la DLL a direcciones 
    concretas en memoria.
\end{itemize}

Esto garantiza que el proceso pueda utilizar código externo (DLLs) sin necesidad 
de incorporarlo estáticamente en su binario, manteniendo modularidad y eficiencia 
en el uso de memoria.








\newpage
\section{Threads}

Los \textit{threads} (hilos) constituyen la unidad básica de ejecución dentro de un proceso. Mientras que un proceso actúa como un contenedor que agrupa recursos (\textit{memory space}, manejadores de objetos, permisos, etc.), un hilo representa el flujo de instrucciones que realmente es ejecutado por la CPU.  
Un proceso puede contener uno o múltiples hilos, cada uno con su propio estado independiente (contador de programa, conjunto de registros, pila), aunque todos comparten el mismo espacio de direcciones del proceso.

Desde la perspectiva de la ciberseguridad y el desarrollo de malware, el control y manipulación de hilos es especialmente relevante. Muchas \textit{Tactics, Techniques and Procedures} (TTPs) se basan en crear nuevos hilos, modificar los existentes o abusar de ellos para ejecutar \textit{payloads} de forma encubierta, minimizando la generación de artefactos monitorizables por soluciones EDR.

\subsection{Thread Environment Block (TEB)}

El \textit{Thread Environment Block} (TEB) es una estructura interna del sistema operativo Windows que almacena información específica del hilo en ejecución. Está situada en el espacio de usuario y es accesible directamente desde el propio hilo, lo que permite interactuar con ella sin necesidad de realizar llamadas a la API Win32. Entre los elementos más relevantes del TEB se incluyen:

\begin{itemize}
    \item Un puntero al \textit{Process Environment Block} (PEB).
    \item Información relacionada con la pila del hilo (dirección base y límite).
    \item La cadena de manejadores de excepciones estructuradas (SEH).
    \item El identificador del hilo (\textit{Thread ID}, TID).
    \item Información sobre la última función Win32 invocada.
    \item Datos asociados al \textit{Thread Local Storage} (TLS).
\end{itemize}

El acceso directo al TEB, sin pasar por funciones exportadas, es una técnica frecuente en \textit{shellcodes} y cargas maliciosas debido a que permite evitar \textit{hooks} a nivel de API y reducir la superficie detectable por mecanismos de seguridad.

\subsection{Walking the TEB}

FALTA DESARROLLO
\newpage
\section*{Flujo de llamadas: User Mode vs Kernel Mode}

\begin{enumerate}
    \item \textbf{User Mode (U) - Aplicación y librerías de alto nivel}
    \begin{itemize}
        \item notepad++.exe: SetLibraryProperty (operaciones internas de la aplicación)
        \item user32.dll: CallWindowProcW, IsWindowUnicode (interacción con interfaz Windows)
        \item KernelBase.dll: WriteFile + 0x8d (función de usuario de alto nivel)
        \item ntdll.dll: ZwWriteFile (interfaz hacia la syscall)
    \end{itemize}

    \item \textbf{Kernel Mode (K) - Operaciones de bajo nivel y hardware}
    \begin{itemize}
        \item FLTMGR.SYS: FltPerformSynchronousIo, FltSetCancelCompletion (filtro de sistema de archivos)
        \item ntoskrnl.exe: IoCallDriver, NtDeviceIoControlFile, NtWriteFile (operaciones sobre drivers y disco)
    \end{itemize}
\end{enumerate}

\noindent
\textbf{Interpretación:} El flujo comienza en User Mode con la aplicación y librerías de alto nivel, y termina en Kernel Mode ejecutando operaciones de escritura en disco con privilegios elevados.

\begin{center}
    \includegraphics[width=0.95\textwidth]{figures/fig-syscall-path.png}
\end{center}