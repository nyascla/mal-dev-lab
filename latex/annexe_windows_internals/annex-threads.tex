\section{Threads}

Los \textit{threads} (hilos) constituyen la unidad básica de ejecución dentro de un proceso. Mientras que un proceso actúa como un contenedor que agrupa recursos (\textit{memory space}, manejadores de objetos, permisos, etc.), un hilo representa el flujo de instrucciones que realmente es ejecutado por la CPU.  
Un proceso puede contener uno o múltiples hilos, cada uno con su propio estado independiente (contador de programa, conjunto de registros, pila), aunque todos comparten el mismo espacio de direcciones del proceso.

Desde la perspectiva de la ciberseguridad y el desarrollo de malware, el control y manipulación de hilos es especialmente relevante. Muchas \textit{Tactics, Techniques and Procedures} (TTPs) se basan en crear nuevos hilos, modificar los existentes o abusar de ellos para ejecutar \textit{payloads} de forma encubierta, minimizando la generación de artefactos monitorizables por soluciones EDR.

\subsection{Thread Environment Block (TEB)}

El \textit{Thread Environment Block} (TEB) es una estructura interna del sistema operativo Windows que almacena información específica del hilo en ejecución. Está situada en el espacio de usuario y es accesible directamente desde el propio hilo, lo que permite interactuar con ella sin necesidad de realizar llamadas a la API Win32. Entre los elementos más relevantes del TEB se incluyen:

\begin{itemize}
    \item Un puntero al \textit{Process Environment Block} (PEB).
    \item Información relacionada con la pila del hilo (dirección base y límite).
    \item La cadena de manejadores de excepciones estructuradas (SEH).
    \item El identificador del hilo (\textit{Thread ID}, TID).
    \item Información sobre la última función Win32 invocada.
    \item Datos asociados al \textit{Thread Local Storage} (TLS).
\end{itemize}

El acceso directo al TEB, sin pasar por funciones exportadas, es una técnica frecuente en \textit{shellcodes} y cargas maliciosas debido a que permite evitar \textit{hooks} a nivel de API y reducir la superficie detectable por mecanismos de seguridad.

\subsection{Walking the TEB}

FALTA DESARROLLO