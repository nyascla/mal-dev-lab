\section{Secciones de un proceso}

El espacio de direcciones de un proceso en memoria se organiza en diferentes secciones.
A continuación, se enumeran las mismas desde las direcciones más altas hasta las más bajas:

\begin{description}[style=nextline, leftmargin=2cm]

    \item[\textbf{Páginas del Kernel}]  
    Ocupan la misma dirección en todos los procesos del sistema y corresponden al kernel.  
    Como todas las páginas de kernel apuntan a los mismos \textit{frames} de memoria física, 
    no se desperdicia espacio cargando el kernel en cada proceso.  
    El acceso se realiza únicamente mediante llamadas al sistema.

    \item[\textbf{Stack (lectura y escritura)}]  
    Estructura de almacenamiento \textit{LIFO}, que crece hacia abajo 
    (desde el final del espacio de direcciones del proceso hacia la sección BSS).  
    Cada función tiene su propio \textit{stack frame}, con variables locales y argumentos.  
    El tamaño suele estar limitado entre 1 y 8 MB, según la arquitectura y el sistema operativo.  
    Su gestión mediante el puntero de pila es muy rápida.

    \item[\textbf{Heap (lectura y escritura)}]  
    Espacio para memoria dinámica gestionada por llamadas como \texttt{malloc}.  
    Crece hacia arriba, desde el final de la BSS hasta el final del espacio de direcciones.  
    Su administración mediante el puntero de \textit{heap} es más lenta que la del \textit{stack}.

    \item[\textbf{BSS y Data (lectura y escritura)}]  
    Almacenan las variables globales y estáticas del programa.  
    La sección \textbf{BSS} contiene las variables no inicializadas,  
    mientras que la sección \textbf{Data} guarda las variables inicializadas.

    \item[\textbf{Text (lectura y ejecución)}]  
    Contiene el código ejecutable del programa, cargado desde el binario en disco.  
    Generalmente está marcada como de solo lectura y ejecución.

\end{description}
