\chapter{Introducción}

\begin{quote}
\textit{¿Porque alguien deberia leer este trabajo?}
\end{quote}

\section{Anatomia de un adversario}

Que es un adverdario

\section{Objetivos del trabajo}

El objetivo de este trabajo es desarrollar un escenario didáctico que permita a
los analistas de ciberseguridad mejorar sus capacidades de deteccion e investigación
mediante la comprensión del proceso completo de creación de un ciberataque. El
resultado final del proyecto será un conjunto de artefactos forenses de los
sistemas infectados, que servirá como base para ejercicios de análisis forense
(blue team).

Para generar dicho recurso, el núcleo del trabajo se centrará en la concepción,
diseño y desarrollo de un ciberataque realista, su ejecución controlada en un
sistema víctima y la documentación íntegra del proceso ofensivo llevado a cabo. 
De este modo, quien reciba los artefactos forenses podrá analizar la infección 
desde el punto de vista del analista forense, pero también interpretar las acciones del atacante, entendiendo mejor
el cómo y el porqué de cada artefacto encontrado.

En definitiva, este trabajo busca aproximar al analista a la mentalidad del
atacante, con el fin de reforzar sus habilidades de detección, interpretación y
respuesta ante amenazas reales.


\chapter{Marco Teórico}

\begin{quote}
\textit{¿Como la industria consigue tener un lenguaje comun para poder transmitir
informacion sobre adversarios de forma efectiva?}
\end{quote}

\section{MITRE ATT\&CK}

\section{CKC}

el framework consiste en dividir un ciberatque en cinco fases:

1. Reconocimiento: investigacion del objetivo
2. Armazenamiento: creacion del payload
3. Entrega: envio del payload al objetivo
4. Explotacion: ejecucion del payload
5. Instalacion: sistema de staging
6. Comando y control: establecimiento del canal de comunicacion
7. Acciones sobre objetivos: acciones finales

\chapter{Threat Intelligence}

\begin{quote}
\textit{¿Como obtener informacion sobre adversarios?}
\end{quote}


\chapter{Panorama Actual de Amenazas}

\begin{quote}
\textit{¿Cuales son las ttp mas utilizadas en la actualidad?}
\end{quote}

Nos centraremos en dos documentos principalemten el m-trends de mandiant
y el 


\chapter{Creando nuestro adversario}


\section{1. Reconocimiento}

\subsection{}

\section{2. Armazenamiento}

\section{3. Entrega}

\section{4. Explotacion}

\section{5. Instalacion}

\section{6. Comando y control}

\section{7. Acciones sobre objetivo}




stage 1 (stub) (Downloader)
stage 2 (Dropper)
payload
payload final